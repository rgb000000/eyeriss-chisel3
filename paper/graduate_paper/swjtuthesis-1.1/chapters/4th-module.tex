\chapter{硬件模块介绍}
\section{Chisel3集成的工具}
Chisel3在进行硬件电路描述的同时也提供了一些常用的工具。参考表\ref{chisel-tool}。
\begin{table}[h] %开始一个表格environment,表格的位置是h,here。  
    \centering
    \caption{Chisel3提供的部分小工具} %显示表格的标题  
    \begin{tabular}{l|l} %设置了每一列的宽度,强制转换。  
    \hline  
    \hline  
    名称 & 描述 \\ %用&来分隔单元格的内容 \\表示进入下一行  
    \hline %画一个横线,下面的就都是一样了,这里一共有4行内容  
    Decoupled() & 用于将数据拓展成标准的ready-valid接口 \\  
    \hline %画一个横线,下面的就都是一样了,这里一共有4行内容  
    Flipped() & 用于接口的翻转,Input翻转为Output,Output翻转为Input \\  
    \hline  
    Queue & 输入为Flipped(Decoupled)输出为Decoupled接口类型的FIFO \\  
    \hline  
    MEM & 同步写、异步读的RAM \\  
    \hline  
    \hline  
    \end{tabular}  
    \label{chisel-tool}
\end{table}  
\begin{figure}[h]
    \centering
    \includegraphics{../pdf/decoupled.pdf}\\
    \caption{Decoupled与Flipped(Decoupled)接口波形图}
    \label{decoupled_w}
\end{figure}

本文大量使用了\emph{Decoupled}类型作为模块之间传递数据的接口。Chisel3中一个\emph{Decoupled}类型接口等效为三个信号,包括\emph{output valid;output bit;input ready},
通过\emph{Flipped}翻转之后的信号等效为\emph{input valid;input bits;input ready},其波形图参考图\ref{decoupled_w},只有当\emph{ready}信号和\emph{valid}信号同时有效时才发生数据的交互。

\section{基于FIFO的可变长移位寄存器设计}
\begin{figure}[h]
    \centering
    \includegraphics[scale=1.1]{../pdf/fifoshift.pdf}\\
    \caption{基于FIFO构造的移位寄存器}
\end{figure}
\begin{table}[h] %开始一个表格environment,表格的位置是h,here。  
    \centering
    \caption{基于FIFO构造的移位寄存器端口说明} %显示表格的标题  
    \begin{tabular}{l|l|c|l} %设置了每一列的宽度,强制转换。  
    \hline  
    \hline  
    端口 & 类型 & 位宽 & 功能 \\ %用&来分隔单元格的内容 \\表示进入下一行  
    \hline %画一个横线,下面的就都是一样了,这里一共有4行内容  
    valid 左 & Input & 1 & 来自生产者的握手信号 \\  
    \hline  
    bits 左 & Input & 1 & 来自生产者的数据 \\  
    \hline  
    ready 左 & Output & w\footnotemark[1] &输出至生产者的握手信号 \\  
    \hline  
    valid 右 & Output & 1 & 输出至消费者的握手信号 \\  
    \hline  
    bits 右 & Output & w & 输出至消费者的数据 \\  
    \hline  
    ready 右 & Input & 1 & 来自消费者的握手信号\\  
    \hline  
    control & Input & 1 & 来自外界的控制内嵌FIFO功能的信号\\  
    \hline  
    \hline  
    \end{tabular}  
\end{table}
\footnotetext[1]{w表示该信号位宽可配置,下文意义相同}
利用FIFO的先入先出特性,配合外部控制信号可以构造一个最大位宽为FIFO深度的移位寄存器。
首先,模块中的Mux将FIFO接口入模块输入信号连接,此时数据被写入FIFO中,当数据输入过程完成之后,Mux将模块内的FIFO输入与输出对接,
此时便形成了一个数据环路,FIFO输出的数据被输入给FIFO的输入,使数据一直保留在FIFO中。该模块工作工程可参考图\ref{shift_k}。
\begin{figure}[h]
    \centering
    \includegraphics[scale=0.8]{../pdf/shift_k.pdf}\\
    \caption{基于FIFO的可变长移位寄存器工作示意图}
    \label{shift_k}
\end{figure}
实际使用时,FIFO深度被配置为256,位宽为16bit。第一阶段,FIFO输入通过Mux与模块的输入对接,接受来自外部的图像或者卷积核信息,第一阶段取数完成后进入第二阶段——计算,此时FIFO的输入与FIFO的输出对接,吐出的数据再次被存入FIFO中,以达到移位寄存器的功能。

需要说明的是,由于FIFO先入先出的特性,该移位构造,读数时每次只能读取最末端的数据,但相比与传统通过寄存器的构造,降低了资源占用率。

    \subsection{BRAM与DRAM选择}
    在Xilinx 7系FPGA中,片上RAM分为BRAM和DRAM两种,其中BRAM为板载RAM,是专门用于实现大容量RAM的部分,而DRAM是通过LUT构造的RAM。实际使用时,大容量RAM通常使用BRAM,但BRAM使用较为笨重,因为其是FPGA中的位置固定,会造成连线传输延时高的问题,同时BRAM时序比较严格,必须使用同步写入、同步读取的方式进行操作。而DRAM相对灵活,由于采用LUT构造,其能出现在FPGA绝大部分位置来最小化传输延时,同时DRAM可以进行异步读取操作,也可以在其后添加寄存器实现同步读取操作。本文所阐述的方案由于需要RAM具有异步读取操作,同时RAM容量小,数量多,因此采用DRAM。

\section{SRAM部分在FPGA中的优化}
\begin{figure}[h]
    \centering
    \includegraphics{../pdf/sram.pdf}\\
    \caption{使用Designware中的SRAM IP核接口图}
\end{figure}
\begin{table}[h] %开始一个表格environment,表格的位置是h,here。  
    \centering
    \caption{调用的Designware中的SRAM IP端口说明} %显示表格的标题  
    \begin{tabular}{l|l|c|l} %设置了每一列的宽度,强制转换。  
    \hline  
    \hline  
    端口 & 类型 & 位宽 & 功能 \\ %用&来分隔单元格的内容 \\表示进入下一行  
    \hline %画一个横线,下面的就都是一样了,这里一共有4行内容  
    clk & Input & 1 & 时钟信号 \\  
    \hline  
    rst\_n & Input & 1 & 复位信号 低有效 \\  
    \hline  
    cs\_n & Input & 1 & 片选信号 低有效 \\  
    \hline  
    wr\_n & Input & 1 & 写使能 低有效\\  
    \hline  
    rd\_addr & Input & $log_2[depth]$ & 读地址 \\  
    \hline  
    wr\_addr & Input & $log_2[depth]$ & 写地址 \\  
    \hline  
    data\_in & Input & w & 输入数据总线\\  
    \hline  
    data\_out & Output & w & 输出数据总线\\  
    \hline  
    \hline  
    \end{tabular}  
\end{table}

Chisel3中构造存储器有两种方式,第一种通过VecInit构造ROM,FPGA实现时是通过线网。第二种通过MEM构造RAM,FPGA实现时是通过寄存器组。本文在设计过程中,首先通过MEM来构造片上RAM,但由于Chisel3 MEM自身的复杂度(其包含了Mask等附加功能),在实现一个6×7的PE阵列时,生成的代码量达到了16万行,通过Vivado综合一次需要90分钟,给调试工作带来了极大的不便利性,同时采用寄存器组的实现方式十分消耗资源。

        \begin{lstlisting}[title=DW\_ram\_r\_w\_s\_dff例化代码参考, frame=shadowbox]
module DW_ram_r_w_s_dff_inst(inst_clk, inst_rst_n,
                             inst_cs_n, inst_wr_n,
                             inst_rd_addr, inst_wr_addr, 
                             inst_data_in, data_out_inst );
parameter data_width = 16;
parameter depth = 256;
parameter rst_mode = 0;
`define bit_width_depth 8 // ceil(log2(depth))
input inst_clk;
input inst_rst_n;
input inst_cs_n;
input inst_wr_n;
input [`bit_width_depth-1 : 0] inst_rd_addr;
input [`bit_width_depth-1 : 0] inst_wr_addr;
input [data_width-1 : 0] inst_data_in;
output [data_width-1 : 0] data_out_inst;
// Instance of DW_ram_r_w_s_dff
DW_ram_r_w_s_dff #(data_width, depth, rst_mode)
U1 (.clk(inst_clk),
.rst_n(inst_rst_n),
.cs_n(inst_cs_n),
.wr_n(inst_wr_n),
.rd_addr(inst_rd_addr),
.wr_addr(inst_wr_addr),
.data_in(inst_data_in),
.data_out(data_out_inst) );
endmodule
        \end{lstlisting}

由于ASIC中RAM通常需要特殊对待,本文在实际开发时,采用了Designware中的\emph{DW\_ram\_r\_w\_s\_dff}ip核。优化之后生成的代码量由16万下降至4万行,Vivado综合实现时间由90分钟下降至5分钟。同时Designware中提供了ip的仿真模型,可以通过Verilator仿真器进行准确的仿真。
在Synplify软件中,对此IP进行实例化,需确保安装了DC,以提供Designware环境,经过综合之后其性能如图\ref{synplify}所示,其综合产生的edf网表文件可以在Vivado中调用。
\begin{figure}[h]
    \centering
    \includegraphics{../pdf/synplify.png}\\
    \caption{Synplify综合结果}
    \label{synplify}
\end{figure}

% \section{基于Xilixn 7Series FPGA片上DSP的高性能乘法器}

\section{PE单元构造}
\begin{figure}[h]
    \centering
    \includegraphics[scale=0.8]{../pdf/data.pdf}\\
    \caption{PE输入数据解析}
    \label{pe_data}
\end{figure}
\begin{figure}[h]
    \centering
    \includegraphics{../pdf/PE.pdf}\\
    \caption{PE结构图}
\end{figure}
\begin{figure}[h]
    \centering
    \includegraphics[scale=0.8]{../pdf/pe_cal.pdf}\\
    \caption{PE计算流程图}
    \label{pe_cal}
\end{figure}
\begin{table}[h] %开始一个表格environment,表格的位置是h,here。  
    \centering
    \caption{PE单元端口说明} %显示表格的标题  
    \begin{tabular}{l|l|c|l} %设置了每一列的宽度,强制转换。  
    \hline  
    \hline  
    端口 & 类型 & 位宽 & 功能 \\ %用&来分隔单元格的内容 \\表示进入下一行  
    \hline %画一个横线,下面的就都是一样了,这里一共有4行内容  
    stateSW & Input & 2 & 状态切换信号 \\
    \hline  
    filter & Flipped-DecoupleIO & w & 输入filter解耦信号 \\
    \hline  
    img & Flipped-DecoupleIO & w & 输入filter解耦信号 \\
    \hline  
    fLen & Input & 8 & 单个filter长度 \\
    \hline  
    fNum & Input & 8 & 输出通道数 \\
    \hline  
    iLen & Input & 8 & 单个img长度 \\
    \hline  
    iNum & Input & 8 & 输入通道数 \\
    \hline  
    relu & Input & 1 & 进行Relu激活高有效 \\
    \hline  
    stateOut & Output & 2 & 输出当前状态 \\
    \hline  
    oSum & Output & w & 输出数据总线 \\
    \hline  
    \hline  
    \end{tabular}  
\end{table}  
利用本章第一节和第二节所述的两个模块,配合相应的控制单元可以十分完美的构造一个用于行静止思想卷积的PE(Process Engine)单元。
其计算流程可以参考图\ref{pe_cal}。
通过对filter、img数据进行排列组合,配合本节设计的PE单元即可实现多卷积核、多图片、多通道的通用性一维卷积操作,提高了本文设计的通用性,减少了控制信号的复杂度。
数据的排列可以参考图\ref{pe_data},其思想为,filter数据按照filter、通道、行内数据的优先级进行排序,img数据按照通道、行内数据、img的优先级进行排序。最后,
可以组成一个糅合了多个、多通道的filter或者img数据,配合图\ref{pe_cal}的计算流程可以实现通用性一维卷积。
当\emph{state=1}时,filter FIFO接受$fLen*fNum*nchannel$个数据,即所有的filter数据。
img FIFO接受$iLen*nchannel$个数据,也即第一个img的第一行的所有通道数据。
计算时,由总计算次数计数器、filter移位次数计数器、img读取新数据次数计数器等计数器互相配合以达到产生正确控制信号的目的,其主要思想为,
首先img第一个数据分别与filter前$fNum$个数据进行相乘,结果分别写入不同SRAM地址,然后img第二个数据分别与filter第二组长度为$fNum$的数据相乘,结果与上次进行累加,这一阶段的移位计算方法可以参考图\ref{pe_cal}第1,2,3行。
不断循环此过程,直到img数据全部计算完成,此时将得到的完整局部计算结果写出,其过程参考图\ref{pe_cal}第4,5行。
img数据计算完成之后,img FIFO需要弹出$nchannel$个数据,并且从外部FIFO存储等长的新img数据,其过程可以参考图\ref{pe_cal}第6行。再次对新的img数据组进行第1,2,3行移位计算。
当img读取新数据次数计数器值等于$iLen - fLen - 1$时,表示此次读取的第一张图片的所有数据均计算完成,需要一次性清空img FIFO并读取新一张img数据,其过程可以参考图\ref{pe_cal}第7行。
不断循环上述过程即可按照计算完成行静止卷积法思想中的filter一行与img一行进行卷积的计算。



\section{用于数据分发的简易NoC设计}
\begin{figure}[h]
    \centering
    \includegraphics{../pdf/node.pdf}\\
    \caption{简易NoC中的一个节点构造}
    \label{node}
\end{figure}
\begin{table}[h] %开始一个表格environment,表格的位置是h,here。  
    \centering
    \caption{简易NoC节点端口说明} %显示表格的标题  
    \begin{tabular}{l|l|c|l} %设置了每一列的宽度,强制转换。  
    \hline  
    \hline  
    端口 & 类型 & 位宽 & 功能 \\ %用&来分隔单元格的内容 \\表示进入下一行  
    \hline %画一个横线,下面的就都是一样了,这里一共有4行内容  
    valid 左 & Input & w & 来自生产者的握手信号 \\
    \hline  
    bits 左 & Input & w & 来自生产者的数据 \\
    \hline  
    ready 左 & Output & 8 & 输出给生产者的握手信号 \\
    \hline  
    valid 右 & Output & 8 & 输出给消费者的握手信号 \\
    \hline  
    bits 右 & Output & 8 & 输出给消费者的数据 \\
    \hline  
    ready 右 & Input & 8 & 来自消费者的握手信号 \\
    \hline  
    \hline
    \end{tabular}  
    \label{node_io}
\end{table}  
\begin{figure}[h]
    \centering
    \includegraphics[scale=0.8]{../pdf/noc.pdf}\\
    \caption{简易NoC示意图}
    \label{nodes}
\end{figure}
NoC(Network-on-Chip)及片上网络,是SoC的一种全新的通信方式,相比于传统总线式通信,基于NoC的系统能更好的适应全局异步局部同步的时钟机制。
但一个完整的NoC系统极其复杂,通信时的数据要经过打包、缓冲、同步等步骤,增加了延时的同时也增加了电路复杂度,不利于缩小电路面积。

实际使用时,本着轻量化的思想,本文所构造的NoC在空间中划分了一个二维节点矩阵,可参考图\ref{nodes},第一个列的节点用于行分发,其余列的节点用于列分发。设计的NoC节点模块内部结构可以参考图\ref{node},其端口描述参考表\ref{node_io}

当信号握手成功后,每个节点都会判断此数据附带的地址是否属于自己,其中用于行分发的节点仅比较数据的行号是否与自身相等,列分发的节点仅比较数据的列号是否与自身相等,因此实际上每个节点仅需要一次比较即可判断是否进行数据转发,减少了资源消耗。
同时考虑到,filter数据的复用特性,约定若列号等于$-1$则每一列都需要转发数据,这种方法成倍的减少了传输filter信号所需要的时间。
同时每个节点中都存在于一个深度为5的FIFO,来减少数据拥塞。

在代码编写时,每个NoC节点都分配了一个唯一的地址(row,col),Chisel3在处理的时候每一个参数不同的模块都会生成一个全新的Module,因此导致了生成代码量较大(一个6×7的阵列代码行数为48517),但这并不影响电路的逻辑功能。

\section{PE阵列生成器}
Chisel3敏捷的一方面就体现在其可以利用Scala中的数据结构来组织硬件电路。在构造PE阵列的时候,本文采用了List[List[]]的数据结构在组织PE阵列,其代码逻辑如下。
            \begin{lstlisting}[title=Chisel 快速构建PE阵列, frame=shadowbox]
val NoC = List[List[Node]]().toBuffer
val pes = List[List[PETesterTop]]().toBuffer
for (i <- Range(0, shape._1)) {
  val tempNoC = List[Node]().toBuffer
  val tempPE = List[PETesterTop]().toBuffer
  for (j <- Range(0, shape._2 + 1)) {
    val node = Module(new Node(j == 0, (i, j), w))
    if (j != 0) { //not row NoC node need to contain a PE
      val pe = Module(new PETesterTop((i, j - 1)))
      pe.io.oSumSRAM.ready := 0.U
      pe.io.stateSW := io.stateSW
      pe.io.peconfig := io.peconfig
      val ds = Module(new dataSwitch())
      ds.io.dataIn <> node.io.dataPackageOut
      pe.io.filter <> ds.io.filter
      pe.io.img <> ds.io.img
      tempPE.append(pe)
    }
    tempNoC.append(node)
  }
  NoC.append(tempNoC.toList)
  pes.append(tempPE.toList)
}
            \end{lstlisting}
\begin{figure}[h]
    \centering
    \includegraphics{../pdf/PEArray.pdf}\\
    \caption{一个3×3PE阵列示意图}
    \label{33arr}
\end{figure}
    程序主要靠两个\emph{for}循环对整个NoC进行行遍历和列遍历,通过\emph{if(col == 0)}来认定一个节点为用于行分发的节点,行分发的节点不需要包含PE计算单元,因此需要特殊对待。
    一个3×3阵列可以参考图\ref{33arr},从图中可以看出,NoC节点会比PE多一列,该列即为用于行分发的节点,不需要包含PE单元。

    \subsection{顶层接口设计}
\begin{table}[h] %开始一个表格environment,表格的位置是h,here。  
    \centering
    \caption{PE阵列端口说明} %显示表格的标题  
    \begin{tabular}{l|l|c|l} %设置了每一列的宽度,强制转换。  
    \hline  
    \hline  
    端口 & 类型 & 位宽 & 功能 \\ %用&来分隔单元格的内容 \\表示进入下一行  
    \hline %画一个横线,下面的就都是一样了,这里一共有4行内容  
    stateSW & Input & 2 & 状态切换信号 \\
    \hline  
    config & Input & w & 传入PE单元计算需要的配置信息 \\
    \hline  
    dataIn & Flipped-DecoupleIO & w & 传入数据包括地址、类型、数据 \\
    \hline  
    oSum & DecoupleIO & w & 输出计算结果 \\
    \hline  
    done & Output & 1 & 计算完成标志高有效 \\
    \hline  
    \hline  
    \end{tabular}  
\end{table}  
    在PE阵列顶层接口设计中,本文为了降低信号复杂度,采用了设计的简易NoC网络传递filter和img数据,同时采用总线的形式传递其余控制信号,例如fLen,fNum等。
    通过这种方式实现控制信号与数据信号的分离,降低了操作信号的复杂度。
    \subsection{计算流程}
\begin{figure}[h]
    \centering
    \includegraphics{../pdf/pearray_k.pdf}\\
    \caption{一个2×2PE阵列计算流程示意图}
    \label{pearray_k}
\end{figure}
一个2×2阵列的计算2×2filter 和 3×3img卷积的示意图可以参考\ref{pearray_k}。
图中每个乘加单元即为一个PE单元,可以接受外部数据,独立完成一次一维多卷积核、多通道、多图片的卷积计算。在一个PE阵列中,同一行的PE单元共享filter数据,img数据按照斜向上的方向进行分发,
及图\ref{pearray_k}左下角的PE单元的img数据与右上角的PE单元的img数据相同。每个PE单元计算完成的结果通过加法树进行叠加,即可得到正确的卷积结果。

\section{本章小结}
本章首先介绍了Chisel3中集成的一些有用的工具,其次详细的介绍了本文所构造的一系列模块的端口说明、运行机制,包括基于FIFO的可变长移位寄存器、SRAM的优化、PE单元等,
其中最为核心的模块是PE计算单元,由于设计时考虑到了CNN卷积层中,通常都包含多卷积核、多通道、多图片的情况,为了减少计算次数,本文构造的PE单元内部控制逻辑较为复杂,
但通过框图的形式介绍了其中核心的逻辑。同时在PE阵列生成器中贴出了Chisel3代码,借助Scala中的数据结构可以十分便利的构造一个二维阵列,代码今有20行即可构造任意规模的阵列,
这在Verilog中是无法轻易实现的。
% \subsection{二级节标题}

% \subsubsection{三级节标题}

% \paragraph{四级节标题}

% \subparagraph{五级节标题}

% \section{脚注}

% Lorem ipsum dolor sit amet, consectetur adipiscing elit, sed do eiusmod tempor
% incididunt ut labore et dolore magna aliqua. Ut enim ad minim veniam, quis
% nostrud exercitation ullamco laboris nisi ut aliquip ex ea commodo consequat.
% Duis aute irure dolor in reprehenderit in voluptate velit esse cillum dolore eu
% fugiat nulla pariatur. Excepteur sint occaecat cupidatat non proident, sunt in
% culpa qui officia deserunt mollit anim id est laborum.
% \footnote{This is a long long long long long long long long long long long long
% long long long long long long long long long long footnote.}
